\chapter{Challenges and Future Work}\label{chp:c-fw}

\section{Challenges}

There were a number of challenges faced with this project. Some challenges derived from the language, others from poor design choices in early versions of MCDTP and MCDTPi. Challenges related to the language stemmed more from the .NET Framework. The discovered asynchrony issue in the cross-platform variant of the .NET Framework was rather obscure. With little documentation on the matter, a significant portion of the investigation involved examining the source code of both the .NET Core Framework and the Common Language Runtime. This was the only way to confirm that I/O Completion Ports were only being used on the Win32 Kernel API.

With respect to MCDTP and MCDTPi design challenges, early iterations had challenges with respect to disk I/O operations blocking socket I/O operations. The early architecture was not modular like MCDTPi is now. Consequently, MCDTPi needed to be restructured to be modular as discussed in Section \ref{sec:mod-arch} and incorporate asynchrony at an architectural level, which is discussed in Section \ref{sec:async-arch}. Additionally, MCDTP was originally designed to use file checksumming to determine which portions of the file needed retransmission. However, this proved to be very inconsistent, partly due to the slowness, and was replaced with the $PacketManagement$ system discussed in Section \ref{sec:pm-sm}.

\section{Future Work}

As mentioned in Chapter \ref{subsec:ftp-hs}, file selection was not included in the design of MCDTP. This feature was left out because it was outside the scope of the project and was left as future work for MCDTP and MCDTPi.

Tests have revealed that the .NET Core Framework does not use I/O Completion Ports on Unix-based operating systems. Testing IOCP is beyond the scope of this project. In order to understand how IOCP will impact performance, a separate study may need to be conducted. Since MCDTP and its implementation are open source, they can be used as candidate software to test the Windows environment. This would provide a direct comparison to Unix-based operating systems.

Asynchronous tasking is an overhead that could be mitigated. Restructuring MCDTPi to more efficiently use asynchrony would reduce the performance impact of TPL. Asynchronous tasks would need to be used on packets in bulk rather than per packet. This is only applicable to MCDTPi. The .NET Core Framework handles sending packets asynchronously and thus is infeasible to modify. Another option would be to implement a custom asynchronous module that tries to mitigate the effects of using TPL. These options would need to be explored in a separate study.

The data transmission phase could be optimized as well. As outlined in the design of MCDTP, a UDP packet can range from $13B \rightarrow 64KB$ in size. However, MCDTPi, is currently limited to $1500B$ in size due to the MTU of the network. Packets larger in size are fragmented to fit this within this limit. This limitation effects bandwidth usage. A path worth exploring to circymvent this would be to prefragment a $64KB$ chunk of data, it would likely need to be less than $64KB$ to account for any headers added to fragments, so that when fragmentation occurred, it would not fragment the data being transferred.
